\documentclass{article}\usepackage[]{graphicx}\usepackage[]{xcolor}
% maxwidth is the original width if it is less than linewidth
% otherwise use linewidth (to make sure the graphics do not exceed the margin)
\makeatletter
\def\maxwidth{ %
  \ifdim\Gin@nat@width>\linewidth
    \linewidth
  \else
    \Gin@nat@width
  \fi
}
\makeatother

\definecolor{fgcolor}{rgb}{0.345, 0.345, 0.345}
\newcommand{\hlnum}[1]{\textcolor[rgb]{0.686,0.059,0.569}{#1}}%
\newcommand{\hlstr}[1]{\textcolor[rgb]{0.192,0.494,0.8}{#1}}%
\newcommand{\hlcom}[1]{\textcolor[rgb]{0.678,0.584,0.686}{\textit{#1}}}%
\newcommand{\hlopt}[1]{\textcolor[rgb]{0,0,0}{#1}}%
\newcommand{\hlstd}[1]{\textcolor[rgb]{0.345,0.345,0.345}{#1}}%
\newcommand{\hlkwa}[1]{\textcolor[rgb]{0.161,0.373,0.58}{\textbf{#1}}}%
\newcommand{\hlkwb}[1]{\textcolor[rgb]{0.69,0.353,0.396}{#1}}%
\newcommand{\hlkwc}[1]{\textcolor[rgb]{0.333,0.667,0.333}{#1}}%
\newcommand{\hlkwd}[1]{\textcolor[rgb]{0.737,0.353,0.396}{\textbf{#1}}}%
\let\hlipl\hlkwb

\usepackage{framed}
\makeatletter
\newenvironment{kframe}{%
 \def\at@end@of@kframe{}%
 \ifinner\ifhmode%
  \def\at@end@of@kframe{\end{minipage}}%
  \begin{minipage}{\columnwidth}%
 \fi\fi%
 \def\FrameCommand##1{\hskip\@totalleftmargin \hskip-\fboxsep
 \colorbox{shadecolor}{##1}\hskip-\fboxsep
     % There is no \\@totalrightmargin, so:
     \hskip-\linewidth \hskip-\@totalleftmargin \hskip\columnwidth}%
 \MakeFramed {\advance\hsize-\width
   \@totalleftmargin\z@ \linewidth\hsize
   \@setminipage}}%
 {\par\unskip\endMakeFramed%
 \at@end@of@kframe}
\makeatother

\definecolor{shadecolor}{rgb}{.97, .97, .97}
\definecolor{messagecolor}{rgb}{0, 0, 0}
\definecolor{warningcolor}{rgb}{1, 0, 1}
\definecolor{errorcolor}{rgb}{1, 0, 0}
\newenvironment{knitrout}{}{} % an empty environment to be redefined in TeX

\usepackage{alltt}
\usepackage[brazil]{babel}
\usepackage{kpfonts}
\usepackage{indentfirst}
\usepackage{parskip}
\usepackage{graphicx}
\usepackage{float}
\graphicspath{ {img/} }
\usepackage{amsmath}
\usepackage{array}

\newenvironment{conditions}
  {\par\vspace{\abovedisplayskip}\noindent\begin{tabular}{>{$}l<{$} @{${}={}$} l}}
  {\end{tabular}\par\vspace{\belowdisplayskip}}


\usepackage{multirow}


\title{A Moeda e o Dinheiro}
\author{FIA Business School}
\IfFileExists{upquote.sty}{\usepackage{upquote}}{}
\begin{document}
\maketitle
\section*{Introdução}

Os \textbf{meios de pagamento} representam todos os haveres com liquidez imediata em poder do público, 
exceto no setor bancário. É uma medida de liquidez do sistema econômico muito utilizado pelo Bacen.\par

Existem muitos ativos (depósitos a prazo, bônus do Bacen, caderneta de poupança, entre outros) que, 
apesar de não serem considerados moeda em sentido estrito, apresentam algumas características da 
moeda em sentido amplo. Costuma-se chamá-los \textit{quase-moeda}, pois podem, sem grandes problemas, ser 
trocados (transformados em moeda).\par

\section*{Identificação dos Meios de Pagamentos}

O \textbf{M4} é o conceito de moeda mais amplo que se tem numa economia e, quando o volume de M4 for baixo, 
significa que há restrições às funções de intermediação financeira do sistema bancário. O aumento 
de M4 em relação a \textbf{M1} costuma ser observado quando se evidenciam processos inflacionários na economia, 
ocorrendo aquilo a que se denomina \textbf{desmonetização}. O contrário (\textbf{monetização}) é verificado quando a 
inflação se reduz, minimizando o custo das pessoas em manter maior volume de moeda M1.\par

\textbf{O M1 não gera rendimento algum para as pessoas, então por que o fazem?}\par

\textbf{Motivo}: negócios ou transações (sincronização entre pagamentos e recebimentos) é explicado pela 
necessidade de as pessoas manterem dinheiro disponível para efetuar seus pagamentos correntes, 
determinados por operações normais e certas. São os vencimentos que ocorrem previamente aos 
recebimentos e os vencimentos dos compromissos financeiros que determinam a demanda por caixa 
pelas pessoas e empresas. Por exemplo, podemos dizer que alguns ajustes nas regras do jogo podem 
ajudar a melhorar o \textbf{fluxo de pagamento}, como a possibilidade de datas diversas para a quitação 
das obrigações, ou também a utilização de cheque para pagamento, ou ainda a utilização de cartão 
de crédito para pagamentos.\par

\textbf{Precaução} (despesas imprevistas e indeterminadas): o público pode acumular dinheiro em mãos e 
evitar aplicações financeiras quando há expectativas negativas na economia e medo de que ocorra 
algum problema com o sistema financeiro nacional (o passado recente da história brasileira dá 
grandes motivos para esse tipo de comportamento das pessoas), pois o público e as firmas precisam 
ter uma certa \textbf{reserva em moeda} para fazer face aos pagamentos imprevistos ou aos atrasos em 
recebimentos esperados.\par

\textbf{Especulação} (oportunidades de negócios): quando existem grandes oportunidades de negócios, as 
pessoas tendem a manter a moeda livre de qualquer embaraço para que possa ser utilizada 
imediatamente. Isso ocorre muito no Brasil, pois temos uma política fiscal que força esse 
comportamento, por exemplo, para a maioria de aplicações financeiras de renda fixa existe um 
imposto chamado IOF (imposto sobre operações financeiras) que tributa o rendimento da aplicação 
caso seja resgatado em curto prazo.\par

\section*{Velocidade de Circulação da Moeda}

Para acompanhar esse movimento, é preciso saber a \textit{velocidade de circulação da moeda}, também 
chamada de velocidade-renda da moeda. Esse indicador aponta o número de vezes que a moeda passa 
de mãos em mãos, num certo período, gerando produção e renda. Essa velocidade  de circulação da 
moeda pode ser calculada para as várias definições de meio de pagamento (M1, M2, M3 e M4). A 
mais utilizada é a M1. Esse valor é praticamente estável em curto prazo e somente tem alterações 
consideráveis quando há:\par

\textbf{mudança de hábito da coletividade} -- quando utilizamos cheque ou cartão de crédito, 
menor é a necessidade de papel-moeda;\par

\textbf{verticalização da economia} -- quando empresas resolvem produzir componentes utilizados na 
produção de seus bens e serviços, não há necessidade de comprar de outra empresa e, consequentemente, 
não há necessidade de ter moedas em mãos.\par

As razões têm um impacto muito lento de alteração da velocidade de circulação da moeda. Então, o que 
pode acelerar ou retardar a velocidade em curto prazo? Dois são os fatores que contribuem para esse salto:\par

\begin{itemize}

\item variações nas taxas de juros do mercado e

\item expectativas de inflação.

\end{itemize}

Uma velocidade de circulação mais lenta revela demanda decrescente por moeda, indicando que as pessoas 
estão reduzindo seus encaixes monetários. Se a demanda por moeda aumentar, é de se esperar que os meios 
de pagamentos circularão mais lentamente.\par

\section*{Efeito Multiplicador de Moeda}

Os bancos comerciais e os bancos múltiplos com carteira comercial possuem a capacidade de criarem moeda 
por meio do chamado \textit{efeito multiplicador da moeda}. Esse procedimento ocorre da seguinte forma: os 
recursos captados por esses bancos são registrados pela contabilidade na conta caixa (ativo) e na conta 
depósito à vista (passivo), ou seja, esse caixa que entrou para o banco pode ser aplicado como empréstimo 
a terceiros que pegará esse dinheiro e colocará em circulação na economia. Em outras palavras, podemos 
dizer que temos um passivo (depósito à vista) criando moeda na economia (empréstimo).\par

Esse ciclo pode repetir-se, se o dinheiro do emprestado voltar a um banco com carteira comercial. 
Para mostrar o mecanismo de expansão monetária (ou seja, da oferta de moeda por meio dos bancos com 
carteiras comerciais), vamos supor que parte das reservas \(r\) dos depósitos à vista feitos no banco esteja 
como reservas (depósito compulsório) no Bacen e parte (\(1-r\)) esteja livre para o banco emprestar ao público 
em geral. Devemos supor que quem recebe empréstimo desse dinheiro (\(1-r\)) livre no Banco decide usar uma 
parte \(c\) e guardar outra (\(1-c\)); esse dinheiro pode ser guardado em um outro banco comercial. Esse novo 
banco reterá parte desse dinheiro como depósito compulsório e emprestará a outras pessoas. Note que esse 
ciclo pode não ter fim. Perceba que tal movimentação forma justamente uma fórmula de \textbf{Progressão Geométrica}, 
portanto podemos dizer que o \textbf{multiplicador monetário} pode ser explicado pela seguinte fórmula:\par

\begin{equation}\label{eq1}
\begin{split}
    M   &=  \frac{ 1 }{ 1 - ( 1 - c )( 1 - r ) }
\end{split}
\end{equation}

Intuitivamente, o efeito multiplicador de moeda varia inversamente em relação às taxas de reservas ou à 
taxa de retenção do público. Quanto mais os bancos forem obrigados a reter moeda em caixa (maior \(r\)), 
menos eles poderão emprestar ao público, e menor será o efeito multiplicador da moeda. Da mesma forma, 
quanto maior for a utilização de moeda pelo público, ou seja, a moeda não é depositada em banco comercial, 
chamado de retenção de moeda pelo público (maior \(c\)) também, menos os Bancos poderão emprestar ao público 
e menor será o efeito multiplicador da moeda. É importante salientar que quanto mais escassa for a moeda 
em poder dos bancos, \textbf{maior será a taxa de juros adotada pelo mercado} (lei da oferta e da procura).\par

A base monetária no passivo equivale a dizer que se trata de todas as exigibilidades monetárias líquidas 
do Bacen perante o público não bancário e os bancos comerciais. Podemos dizer que a base monetária é 
representada pela diferença entre o ativo total (aplicações) e os recursos não monetários obtidos pelo Bacen.\par

Por exemplo, quando se verifica um resultado líquido positivo no fluxo de entradas e saídas de moedas 
estrangeiras no Brasil, o saldo da base monetária expande-se pela maior oferta de moeda nacional convertida. 
O Bacen pode decidir manter a base monetária em mesmos níveis ofertando títulos públicos federais. Agora, se 
o Bacen emitir moeda ou puser em circulação as que estão em sua posse para reaver os títulos, como 
consequência aumentará a base monetária em circulação.













\end{document}
