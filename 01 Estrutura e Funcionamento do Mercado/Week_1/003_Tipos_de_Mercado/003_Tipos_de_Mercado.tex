\documentclass{article}\usepackage[]{graphicx}\usepackage[]{xcolor}
% maxwidth is the original width if it is less than linewidth
% otherwise use linewidth (to make sure the graphics do not exceed the margin)
\makeatletter
\def\maxwidth{ %
  \ifdim\Gin@nat@width>\linewidth
    \linewidth
  \else
    \Gin@nat@width
  \fi
}
\makeatother

\definecolor{fgcolor}{rgb}{0.345, 0.345, 0.345}
\newcommand{\hlnum}[1]{\textcolor[rgb]{0.686,0.059,0.569}{#1}}%
\newcommand{\hlstr}[1]{\textcolor[rgb]{0.192,0.494,0.8}{#1}}%
\newcommand{\hlcom}[1]{\textcolor[rgb]{0.678,0.584,0.686}{\textit{#1}}}%
\newcommand{\hlopt}[1]{\textcolor[rgb]{0,0,0}{#1}}%
\newcommand{\hlstd}[1]{\textcolor[rgb]{0.345,0.345,0.345}{#1}}%
\newcommand{\hlkwa}[1]{\textcolor[rgb]{0.161,0.373,0.58}{\textbf{#1}}}%
\newcommand{\hlkwb}[1]{\textcolor[rgb]{0.69,0.353,0.396}{#1}}%
\newcommand{\hlkwc}[1]{\textcolor[rgb]{0.333,0.667,0.333}{#1}}%
\newcommand{\hlkwd}[1]{\textcolor[rgb]{0.737,0.353,0.396}{\textbf{#1}}}%
\let\hlipl\hlkwb

\usepackage{framed}
\makeatletter
\newenvironment{kframe}{%
 \def\at@end@of@kframe{}%
 \ifinner\ifhmode%
  \def\at@end@of@kframe{\end{minipage}}%
  \begin{minipage}{\columnwidth}%
 \fi\fi%
 \def\FrameCommand##1{\hskip\@totalleftmargin \hskip-\fboxsep
 \colorbox{shadecolor}{##1}\hskip-\fboxsep
     % There is no \\@totalrightmargin, so:
     \hskip-\linewidth \hskip-\@totalleftmargin \hskip\columnwidth}%
 \MakeFramed {\advance\hsize-\width
   \@totalleftmargin\z@ \linewidth\hsize
   \@setminipage}}%
 {\par\unskip\endMakeFramed%
 \at@end@of@kframe}
\makeatother

\definecolor{shadecolor}{rgb}{.97, .97, .97}
\definecolor{messagecolor}{rgb}{0, 0, 0}
\definecolor{warningcolor}{rgb}{1, 0, 1}
\definecolor{errorcolor}{rgb}{1, 0, 0}
\newenvironment{knitrout}{}{} % an empty environment to be redefined in TeX

\usepackage{alltt}
\usepackage[brazil]{babel}
\usepackage{kpfonts}
\usepackage{indentfirst}
\usepackage{parskip}
\usepackage{graphicx}
\usepackage{float}
\graphicspath{ {img/} }
\usepackage{amsmath}
\usepackage{array}
\usepackage{csquotes}

\newenvironment{conditions}
  {\par\vspace{\abovedisplayskip}\noindent\begin{tabular}{>{$}l<{$} @{${}={}$} l}}
  {\end{tabular}\par\vspace{\belowdisplayskip}}


\usepackage{multirow}


\title{Tipos de Mercado}
\author{FIA Business School}
\IfFileExists{upquote.sty}{\usepackage{upquote}}{}
\begin{document}
\maketitle

\section*{Mercado Monetário}

O mercado monetário encontra-se estruturado visando ao controle da liquidez monetária da economia 
e das taxas de juros fixadas pelas autoridades monetárias. Nesse mercado, as instituições 
financeiras costumam suprir seus desencaixes financeiros eventuais, o governo financia suas 
necessidades de caixa e realiza a rolagem de sua dívida. Os papéis são negociados nesse mercado 
tendo como parâmetro de referência a \textbf{taxa de juros}, que se constitui em sua mais importante moeda 
de transação. Os papéis que lastreiam as operações do mercado monetário caracterizam-se pelos 
reduzidos prazos de resgate e alta liquidez. É um mercado de curto e curtíssimo prazos (duração de 
até um ano), sendo ainda responsável pela formação das taxas de juros da economia (taxa Selic e taxa DI).\par

São negociados, principalmente, os papéis emitidos pelo \textbf{Tesouro Nacional}, com o objetivo de 
financiar o orçamento público (ex.: NTN – Notas do Tesouro Nacional; LTN – Letras do Tesouro 
Nacional), além de diversos títulos públicos emitidos pelos Estados e Municípios.\par

São ainda negociados no mercado monetário os \textbf{certificados de depósitos interfinanceiros} (CDI), 
exclusivamente entre instituições financeiras, e títulos de emissão privada, como o \textbf{certificado 
de depósito bancário} (CDB) e debêntures.\par

O mercado monetário é essencial para o estabelecimento do nível de liquidez da economia, 
controlando e regulando o fluxo de moeda convencional (papel-moeda) e de moeda escritural 
(depósitos à vista nos bancos comerciais). Para adequar o volume de moeda com o objetivo de se 
manter a liquidez da economia, a autoridade monetária (Banco Central) atua no mercado financeiro 
disponibilizando ou retirando recursos da economia.\par

Para aumentar a oferta de moeda, o Banco Central compra títulos no mercado, injetando recursos. 
Ao contrário, para reduzir a liquidez, o Banco Central vende títulos para os investidores de 
mercado, retirando recursos da economia.\par

\section*{Mercado de Crédito}

O mercado de crédito visa fundamentalmente suprir as necessidades de caixa de curto e médio prazos 
dos vários agentes econômicos, seja por meio da concessão de créditos às pessoas físicas, seja por 
empréstimos e financiamentos às empresas.\par

As operações desse mercado, dentro de uma política de especialização do Sistema Financeiro Nacional, 
são tipicamente realizadas por instituições financeiras bancárias (bancos comerciais e múltiplos). 
As atividades dos bancos, que visam principalmente reforçar o volume de captação de recursos, têm 
evoluído para um processo de diversificação de produtos financeiros e na área de serviços prestados.\par

Muitas vezes, são também inclusas no âmbito do mercado de crédito as operações de financiamento de 
bens de consumo duráveis realizadas pelas sociedades financeiras. Nessa estrutura, a atuação do 
mercado torna-se mais abrangente, provendo recursos em médio prazo, por meio de instituições 
financeiras não bancárias, aos consumidores de bens de consumo.\par

\textbf{Captação de recursos de instituições financeiras} (em depósito à prazo e a vista): neste caso, o 
Bacen é responsável por disciplinar, controlar e fiscalizar a emissão, intermediação e níveis de 
risco, em:

\begin{itemize}
  
  \item emissão de CDB, RDB (Recibo de Depósito Bancário), Letras e Cédulas emitidas por Bancos;
  
  \item captação de recursos de Poupança e
  
  \item depósito de recursos em Conta Corrente (Depósito à vista);
  
\end{itemize}

\textbf{Operações de crédito de instituições financeiras} (inclusive \textit{factoring}): neste caso, o Bacen 
é responsável por disciplinar e fiscalizar a oferta, prazo e exposições aos riscos, em:

\begin{itemize}

  \item empréstimos;
  
  \item financiamentos;
  
  \item descontos comerciais;
  
  \item crédito imobiliário e
  
  \item arrendamento mercantil (\textit{leasing}).
  
\end{itemize}

\section*{Mercado de Capitais}

O mercado de capitais assume papel dos mais relevantes no processo de desenvolvimento econômico. 
É o grande municiador de recursos permanentes para a economia, em virtude da ligação que efetua 
entre os que têm capacidade de poupança, ou seja, os investidores, e aqueles carentes de recursos 
de longo prazo, ou seja, os que apresentam déficit de investimento.\par

O mercado de capitais está estruturado para suprir as necessidades de investimentos dos agentes 
econômicos, por meio de diversas modalidades de financiamentos de médio e longo prazos para 
capital de giro e capital fixo. É constituído pelas instituições financeiras não bancárias, 
instituições componentes do sistema de poupança e empréstimo (SBPE) e diversas instituições auxiliares.\par

O mercado de capitais oferece também financiamentos com prazo indeterminado, como as operações 
que envolvem a emissão e subscrição de ações.\par

\section*{Mercado Cambial}

O mercado cambial é o segmento financeiro em que ocorrem operações de compras e vendas de moedas 
internacionais conversíveis, ou seja, em que se verificam conversões de moeda nacional em estrangeiras 
e vice-versa. Esse mercado reúne todos os agentes econômicos que tenham motivos para realizar 
transações com o exterior, como operadores de comércio internacional, instituições financeiras, 
investidores e bancos centrais, que tenham necessidades de realizar exportações e importações, 
pagamentos de dividendos, juros e principal de dívidas, royalties e recebimentos de capitais e outros valores.\par

No Brasil, a política cambial é definida pelo CMN e executada pelo Bacen.\par

A \textbf{taxa de câmbio} revela a relação entre uma moeda e outra moeda, ou seja, as unidades de uma moeda 
em comparação com a outra. Por exemplo, a moeda brasileira (R\$) pode ser expressa em relação ao 
dólar (US\$) em R\$ 5,40 por US\$ 1.00.\par

O Banco Central, de forma mais específica, atua nesse mercado cambial visando principalmente a 
controlar as \textbf{reservas cambiais} da economia e a manter o valor da moeda nacional em relação a outras 
moedas internacionais.\par

No Brasil, o mercado de câmbio é regulamentado e fiscalizado pelo \textbf{Banco Central}. As operações de 
câmbio são realizadas entre os agentes autorizados (bancos, corretoras e distribuidoras, caixas 
econômicas, agências de turismo etc.)  e entre esses agentes e seus clientes.\par

Exceto as operações de viagens internacionais e outras, bem específicas, as demais operações de câmbio 
no Brasil são, em sua maioria, liquidadas pela emissão de ordem de pagamento.\par

A demanda por moeda estrangeira está refletida nos importadores, investidores internacionais, 
devedores que desejam amortizar seus compromissos com credores estrangeiros, empresas multinacionais 
que necessitem remeter capitais e dividendos etc. No grupo de vendedores de moedas estrangeiras, 
incluem-se os exportadores, os tomadores de empréstimos, os turistas que deixam o país, entre outros.\par

As operações cambiais processam-se basicamente por meio de operadores (corretores) de câmbio, que são 
especialistas vinculados às instituições financeiras na função de transacionar divisas, e as corretoras 
de câmbio, que atuam como intermediários entre os operadores e os agentes econômicos interessados em 
comprar ou vender moedas. O corretor de câmbio intervém nas operações cambiais aproximando as partes 
interessadas em negociar divisas e municiando os participantes com importantes informações relativas 
às negociações e taxas de mercado.\par

O mercado de câmbio no Brasil encontra-se atualmente segmentado em:

\begin{enumerate}

  \item mercado de câmbio de taxas livres – \textbf{câmbio comercial} e
  
  \item mercado de câmbio de taxas flutuantes – \textbf{câmbio turismo}.
  
\end{enumerate}

As \textbf{taxas de câmbio comercial} (taxas livres) cobrem as operações de importações e exportações, 
pagamentos internacionais de juros e dividendos, empréstimos externos, investimentos de capital etc. 
As taxas flutuantes incluem as negociações de moeda estrangeira para operações de turismo e 
demais despesas relacionadas.\par

\textbf{Taxa de câmbio spot} ocorre quando o pagamento pela moeda estrangeira e a respectiva entrega da 
moeda nacional são realizados à vista. \textbf{Taxa de câmbio forward}, por sua vez, acontece quando a 
entrega e pagamento se dão numa data futura, geralmente em prazo não inferior a um mês.\par

A taxa spot é liquidada na prática em até 2 dias úteis (D+2)  e a taxa forward (ou taxa futura) 
é utilizada para liquidação em prazo superior a 2 dias úteis.\par

Nas operações \textit{forward}, a taxa de câmbio é estabelecida livremente pelas partes contratantes no 
momento da negociação. O principal uso desse mercado a prazo é o de criar proteção contra o risco 
cambial em operações realizadas com moeda estrangeira para liquidação futura.\par

\section*{Política Cambial}

A administração das taxas de câmbio influencia diretamente as transações internacionais de um país, 
como remessas de capitais, empréstimos estrangeiros, exportações, importações etc. Uma expansão nas 
exportações, por exemplo, pode ocasionar um aumento da base monetária, na medida em que as moedas 
estrangeiras recebidas nas vendas são transformadas em moeda nacional que entra na economia.\par

Esse controle da paridade cambial define o valor com que um país aceita negociar sua moeda, 
tanto para venda (compra de moeda estrangeira) quanto para compra (venda de moeda estrangeira).\par

\section*{Cupom Cambial}

É a remuneração efetiva dos valores de uma moeda estrangeira, convertidos em reais e aplicados no 
mercado financeiro brasileiro. O mais comum no mercado financeiro brasileiro é utilizar o dólar como 
moeda de referência a ser convertida em reais, mas o cálculo pode ser feito para qualquer moeda estrangeira.\par

O valor do cupom cambial é obtido pela relação (diferença) entre as taxas de juros remuneratórios dos 
reais (em títulos públicos, por exemplo) e a variação esperada na taxa de câmbio (desvalorização ou 
valorização do real), relativamente a uma dada moeda, pelo período de tempo da aplicação.\par

O valor assim obtido deve ser comparado pelo investidor com as taxas de juros remuneratórios do 
país de origem da moeda, levando-se também em comparação o risco de se aplicar em moeda estrangeira 
no mercado brasileiro e aspectos tributários. Assim, suponha-se o seguinte exemplo:

\begin{itemize}

  \item taxa de juros remuneratórios no Brasil (Títulos Públicos) = 18,75\% a.a.;
  
  \item expectativa de variação da moeda em questão = 6\% a.a. (desvalorização do real);
  
  \item período de aplicação = 1 ano;
  
  \item Valor Bruto do Cupom = 1,1875/1,06 = 1,1203 ou 12,03\% a.a.
  
  \item taxa de juros remuneratórios no país de origem da moeda = 6,00\% a.a.
  
\end{itemize}

O investimento, em princípio, torna-se interessante para o investidor, não se levando em 
consideração o risco-país e despesas e custos tributários e operacionais.\par

\section*{Reservas Internacionais}

É o total de moeda estrangeira (principalmente dólares, no caso brasileiro) mantido pelo Banco 
Central (BC), disponível para uso imediato.\par

As reservas internacionais têm origem nos superávits do \textbf{balanço de pagamentos}. Toda vez que há 
uma entrada de moeda estrangeira, o BC realiza o câmbio, ficando com os dólares e pagando os 
exportadores em reais.  Quando há mais entradas de dólares que saídas, o BC acumula reservas; 
inversamente, quando o país é deficitário, há uma saída de divisas que o BC cobre fazendo uso 
das reservas acumuladas.\par

Além da função de cobrir os eventuais déficits nas contas externas, as reservas internacionais 
também podem ser usadas para evitar ataques especulativos contra a moeda. Assim, quando 
especuladores do mercado financeiro tentam provocar fortes altas ou baixas do dólar no mercado, 
o BC pode usar as reservas para neutralizar esses movimentos.\par

Há dois critérios para o cálculo do volume de reservas internacionais. O descrito acima é conhecido 
pela expressão \enquote{reservas internacionais conceito caixa}. Mas, além desse, também existe o conceito 
de liquidez internacional,que, além dos valores acima, também considera títulos em dólar e outros 
recursos de médio e longo prazos em poder do BC. O BC deixou de divulgar a série no conceito caixa 
em dezembro de 2001, permanecendo apenas com a \textbf{série de liquidez internacional}.\par

\section*{Regimes de Taxa de Câmbio}

O regime de câmbio, a partir da interferência governamental nas forças de mercado de transações em 
moedas estrangeiras, é chamado de livre ou flutuante, quando não há restrições à entrada e saída de 
divisas do país, e de controlado ou fixo quando ocorre o inverso.\par

No caso do câmbio flutuante, as autoridades monetárias interferem somente para impedir especulações 
que possam ser prejudiciais ao mercado interno, possam comprometer suas reservas cambiais ou por 
motivos de segurança nacional. A intervenção é caracterizada como medida de política econômica 
dirigida especificamente às atividades ameaçadas e não como atitude decorrente de política cambial.\par

Com relação ao regime de controle de câmbio, normalmente essa prática é adotada por países que têm 
problemas de balanço de pagamentos, cujo equilíbrio os governos visam restabelecer. O objetivo é gerar 
superávits para aumentar as reservas cambiais do país e utilizá-las na aquisição dos produtos que exercem 
maiores efeitos na promoção do crescimento econômico. O controle do câmbio é, em geral, utilizado como 
arma de política econômica e  frequentemente acompanhado por medidas de controle do comércio exterior, 
como o estabelecimento de cotas de importação e restrições a determinadas mercadorias ou países.\par

\section*{Contas Externas}

Desde janeiro de 2001, o Banco Central divulga as contas externas brasileiras, seguindo a metodologia 
recomendada pelo Fundo Monetário Internacional (FMI) por meio do \textbf{Manual do Balanço de Pagamentos}. 
Esse manual define as normas internacionais para a apuração das contas externas de forma integrada, 
englobando os dados de fluxos (o Balanço de Pagamentos propriamente dito) e os dados de estoque de 
ativos e passivos financeiros internacionais do país.\par

O Balanço de Pagamentos registra os valores de todas as transações internacionais efetuadas por um país, 
dentro de um determinado período de tempo. Sua estrutura é apresentada em dois grandes grupos de contas: 
a \textbf{Conta Corrente} e a \textbf{Conta Capital e Financeira}.\par

A Conta Corrente agrega:

\begin{itemize}

  \item a \textbf{Balança Comercial} --- registra o saldo apurado das exportações menos as importações. Essas 
        transações são fixadas por seu valor \textit{Free On Board} (FOB), ou seja, pelo valor de embarque das 
        mercadorias, não incluindo fretes e seguros;
        
  \item a \textbf{Balança de Serviços e Rendas} --- corresponde aos vários pagamentos e recebimentos realizados 
        entre residentes no país com o resto do mundo, relativos a seguros, fretes, royalties e assistência 
        técnica, viagens internacionais, lucros e dividendos, lucros reinvestidos, gastos governamentais 
        com embaixadas e organismos internacionais, entre outros;
  
  \item as \textbf{Transferências Unilaterais Correntes Líquidas} --- referem-se às doações de mercadorias e 
        assistência, remessa de imigrantes, entre outros.

\end{itemize}

O \textbf{Saldo em Conta Corrente} representa o somatório dos saldos das contas acima e revela se o país está exportando 
ou importando poupança. Um saldo negativo mostra que o país está financiando seus investimentos internos por meio de 
poupança externa. Ou seja, a poupança interna é insuficiente, tendo que recorrer a recursos externos para viabilizar 
sua formação de capital. Um superávit demonstra que o país está enviando para o exterior excedentes de poupança 
interna, colaborando para financiar investimentos do restante do mundo.\par

A Conta Capital e Financeira agrega:

\begin{itemize}

  \item os \textbf{Investimentos Diretos} e em Carteira de Estrangeiros no País e de Brasileiros no Exterior;
  
  \item os \textbf{Empréstimos e Financiamentos} Contratados pelos Diversos Setores da Economia;
  
  \item os \textbf{Reinvestimentos e Repatriações de Investimentos};
  
  \item as \textbf{Amortizações de Dívidas};
  
  \item as \textbf{Operações com Derivativos} e \textbf{Outros Investimentos}.
  
\end{itemize}

A soma dos resultados das \textbf{Contas Correntes} e \textbf{Capital e Financeira} constitui o resultado global 
do \textbf{Balanço de Pagamentos} que, por definição, é igual à variação das reservas internacionais no conceito liquidez 
internacional. Erros e omissões podem dar margem a uma discrepância entre os dois fluxos, que é devidamente 
registrada no Balanço de Pagamentos.

\end{document}
