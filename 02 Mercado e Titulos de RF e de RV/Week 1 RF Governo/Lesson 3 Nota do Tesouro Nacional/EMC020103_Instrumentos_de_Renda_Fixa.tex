\documentclass{article}\usepackage[]{graphicx}\usepackage[]{xcolor}
% maxwidth is the original width if it is less than linewidth
% otherwise use linewidth (to make sure the graphics do not exceed the margin)
\makeatletter
\def\maxwidth{ %
  \ifdim\Gin@nat@width>\linewidth
    \linewidth
  \else
    \Gin@nat@width
  \fi
}
\makeatother

\definecolor{fgcolor}{rgb}{0.345, 0.345, 0.345}
\newcommand{\hlnum}[1]{\textcolor[rgb]{0.686,0.059,0.569}{#1}}%
\newcommand{\hlstr}[1]{\textcolor[rgb]{0.192,0.494,0.8}{#1}}%
\newcommand{\hlcom}[1]{\textcolor[rgb]{0.678,0.584,0.686}{\textit{#1}}}%
\newcommand{\hlopt}[1]{\textcolor[rgb]{0,0,0}{#1}}%
\newcommand{\hlstd}[1]{\textcolor[rgb]{0.345,0.345,0.345}{#1}}%
\newcommand{\hlkwa}[1]{\textcolor[rgb]{0.161,0.373,0.58}{\textbf{#1}}}%
\newcommand{\hlkwb}[1]{\textcolor[rgb]{0.69,0.353,0.396}{#1}}%
\newcommand{\hlkwc}[1]{\textcolor[rgb]{0.333,0.667,0.333}{#1}}%
\newcommand{\hlkwd}[1]{\textcolor[rgb]{0.737,0.353,0.396}{\textbf{#1}}}%
\let\hlipl\hlkwb

\usepackage{framed}
\makeatletter
\newenvironment{kframe}{%
 \def\at@end@of@kframe{}%
 \ifinner\ifhmode%
  \def\at@end@of@kframe{\end{minipage}}%
  \begin{minipage}{\columnwidth}%
 \fi\fi%
 \def\FrameCommand##1{\hskip\@totalleftmargin \hskip-\fboxsep
 \colorbox{shadecolor}{##1}\hskip-\fboxsep
     % There is no \\@totalrightmargin, so:
     \hskip-\linewidth \hskip-\@totalleftmargin \hskip\columnwidth}%
 \MakeFramed {\advance\hsize-\width
   \@totalleftmargin\z@ \linewidth\hsize
   \@setminipage}}%
 {\par\unskip\endMakeFramed%
 \at@end@of@kframe}
\makeatother

\definecolor{shadecolor}{rgb}{.97, .97, .97}
\definecolor{messagecolor}{rgb}{0, 0, 0}
\definecolor{warningcolor}{rgb}{1, 0, 1}
\definecolor{errorcolor}{rgb}{1, 0, 0}
\newenvironment{knitrout}{}{} % an empty environment to be redefined in TeX

\usepackage{alltt}

\usepackage[autostyle]{csquotes}
\usepackage[brazil]{babel}
\usepackage[T1]{fontenc}
\usepackage[utf8]{inputenc}
\usepackage{amsmath}
\usepackage{array}
\usepackage{booktabs,caption}
\usepackage{float}
\usepackage{float}
\usepackage{geometry}
\usepackage{graphicx}
\usepackage{indentfirst}
\usepackage{kpfonts}
\usepackage{multirow}
\usepackage{parskip}
\usepackage{siunitx}
\usepackage{textcomp}



\geometry{
paperwidth    = 210 mm,
paperheight   = 297 mm,
layoutwidth   = 210 mm,
layoutheight  = 297 mm,
layouthoffset = 5 mm,
layoutvoffset = 5 mm,
textwidth     = 150 mm,
textheight    = 200 mm,
includehead   = false,
includefoot   = false,
lmargin       = 25 mm,
rmargin       = 25 mm,
tmargin       = 25 mm,
bmargin       = 25 mm,
heightrounded
}


\graphicspath{ {img/} }

\newenvironment{conditions}
  {\par\vspace{\abovedisplayskip}\noindent\begin{tabular}{>{$}l<{$} @{${}={}$} l}}
  {\end{tabular}\par\vspace{\belowdisplayskip}}


\newcommand{\brl}{R\$}




\sisetup{
round-mode = places,
round-precision = 2,
output-decimal-marker={.},
group-separator ={,},
group-minimum-digits = 3
}







\title{Nota do Tesouro Nacional}
\author{FIA Business School}
\IfFileExists{upquote.sty}{\usepackage{upquote}}{}
\begin{document}

\maketitle

\section*{Introdução}

\section*{NTN-B}

\subsection*{Exemplo}

\section*{NTN-C}

\section*{NTN-F}

\subsection*{Exemplo}

\end{document}

% \begin{itemize}
% 
%   \item \textbf{LOREM IPSUM}. Lorem ipsum dolor sit amet, consectetur adipiscing elit, 
%   sed do eiusmod tempor incididunt ut labore et dolore magna aliqua.
% 
%   \item \textbf{LOREM IPSUM}. Lorem ipsum dolor sit amet, consectetur adipiscing elit, 
%   sed do eiusmod tempor incididunt ut labore et dolore magna aliqua.
% 
% \end{itemize}

% \begin{table}[H]
% \centering
% \caption{Diferentes tipos de Operadores.}
% \resizebox{\textwidth}{!}{%
% \begin{tabular}{ |p{2cm}|p{2cm}|  }
%  \hline
%  \multicolumn{2}{|c|}{Subsistema Operativo}\\
%  \hline
%  \textbf{Espécie}                                                               & \textbf{Operadores (atividades)}\\
%  \hline
%  \multicolumn{1}{|l|}{Instituição Financeira Captadora de Depósito à Vista}     & \multicolumn{1}{l|}{Bancos Comerciais}\\
%  \hline
%  \multicolumn{1}{|l|}{Cooperativas de Crédito}                                  & \multicolumn{1}{l|}{}\\
%  \multicolumn{1}{|l|}{Agentes Especiais:}                                       & \multicolumn{1}{l|}{}\\
%  \multicolumn{1}{|l|}{$\bullet$ Banco do Brasil}                                & \multicolumn{1}{l|}{}\\
%  \multicolumn{1}{|l|}{$\bullet$ Banco do Nordeste}                              & \multicolumn{1}{l|}{}\\
%  \multicolumn{1}{|l|}{$\bullet$ Banco da Amazônia}                              & \multicolumn{1}{l|}{}\\
%  \multicolumn{1}{|l|}{$\bullet$ Caixa Econônica Federal}                        & \multicolumn{1}{l|}{}\\
%  \hline
%  \multicolumn{1}{|l|}{\textbf{Demais Instituições Financeiras}}                 & \multicolumn{1}{l|}{Banco de Investimento}\\
%  \multicolumn{1}{|l|}{Banco de Câmbio}                                          & \multicolumn{1}{l|}{}\\
%  \multicolumn{1}{|l|}{Associações de Poupança e Empréstimo}                     & \multicolumn{1}{l|}{}\\
%  \multicolumn{1}{|l|}{Sociedade de Crédito Imobiliário}                         & \multicolumn{1}{l|}{}\\
%  \multicolumn{1}{|l|}{Companhias Hipotecárias}                                  & \multicolumn{1}{l|}{}\\
%  \multicolumn{1}{|l|}{Sociedade de Crédito, Financiamento e Investimento}       & \multicolumn{1}{l|}{}\\
%  \multicolumn{1}{|l|}{Sociedade de Crédito ao Microempreendedor}                & \multicolumn{1}{l|}{}\\
%  \multicolumn{1}{|l|}{Agência de Fomento (governo Estadual)}                    & \multicolumn{1}{l|}{}\\
%  \multicolumn{1}{|l|}{Banco de Desenvolvimento (governo Estadual)}              & \multicolumn{1}{l|}{}\\
%  \multicolumn{1}{|l|}{BNDES (Agente Especial do governo Federal)}               & \multicolumn{1}{l|}{}\\
%  \hline
%  \multicolumn{1}{|l|}{\textbf{Banco Múltiplo}}                                & \multicolumn{1}{l|}{Banco Múltiplo}\\
%  \hline
%  \multicolumn{1}{|l|}{\textbf{Outros Intermediários Financeiros}} & \multicolumn{1}{l|}{Sociedade de Arrendamento Mercantil (Leasing)}\\
%  \multicolumn{1}{|l|}{Administradoras de Consórcios}                            & \multicolumn{1}{l|}{}\\
%  \multicolumn{1}{|l|}{Sociedade Corretoras de Câmbio}                           & \multicolumn{1}{l|}{}\\
%  \multicolumn{1}{|l|}{Sociedade Corretoras de Títulos e Valores Mobiliários}    & \multicolumn{1}{l|}{}\\
%  \multicolumn{1}{|l|}{Sociedade Distribuidora de Títulos e Valores Mobiliários} & \multicolumn{1}{l|}{}\\
%  \multicolumn{1}{|l|}{\textbf{Auxiliares Financeiros}}                          & \multicolumn{1}{l|}{Bolsa de Valores}\\
%  \multicolumn{1}{|l|}{Bolsa de Mercadorias}                                     & \multicolumn{1}{l|}{}\\
%  \multicolumn{1}{|l|}{Bolsa de Futuros}                                         & \multicolumn{1}{l|}{}\\
%  \multicolumn{1}{|l|}{Sistemas e Câmaras de Liquidação e Custódia}              & \multicolumn{1}{l|}{}\\
%  \hline
% \end{tabular}}
% \end{table}

Ofertas públicas por leilão, ofertas públicas sem a realização de leilão e emissões 
para atender as necessidades específicas determinadas em lei.

Quando acreditar na redução das taxas de juros, por ser um título prefixado, 
adquirido quando a taxa de juros estava em patamares mais altos.

o preço pelo qual o título está sendo negociado. É o valor presente dos fluxos 
de caixa futuros, descontadas as taxas de juros do mercado. 

são títulos prefixados. 
